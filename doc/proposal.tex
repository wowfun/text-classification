\documentclass[sigplan]{acmart}
\usepackage{natbib} 
%\usepackage{gbt7714}

\usepackage{xeCJK}
\setCJKmainfont{Songti SC}

\settopmatter{printacmref=false} 
\pagestyle{plain} 
\renewcommand\footnotetextcopyrightpermission[1]{}


\author {操海洲}
\email{haggi.cho@outlook.com}
\affiliation{%
  \institution{202028016029011}
}

\author {杨旭锐}
\email{yangxurui20g@ict.ac.cn}
\affiliation{%
  \institution{2020E8013282023}
}

\author {王怡宁}
\email{wyn_cnic@163.com}
\affiliation{%
  \institution{2020E8016082028}
}

\author {祝菲}
\email{zhufei@iie.ac.cn}
\affiliation{%
  \institution{202028018629033}
}

\author {陆峰}
\email{lukunjie@live.com}
\affiliation{%
  \institution{2020E8013282001}
}


\begin{document}
		\title{中文新闻文本分类研究}

		\maketitle
		
		\section{问题描述}
		\hspace{2.0em}\setlength{\parindent}{2em}文本分类是自然语言处理领域(NLP)中一个经典问题,旨在对一些如句子、段落和文档等文本单元打上标签或分类。其方法目前广泛应用于情感分析、垃圾邮件检测、新闻分类、内容审核等场景中。
		\par 文本分类方法在新闻领域有着重要的应用场景,它能够实现对新闻信息的自动归类。近年来,随着互联网的广泛普及,新式词汇和文体的出现速度明显加快,给文本分类带来了新的巨大挑战。同时,网络使得信息的传播渠道得到了极大的扩宽,传播速度迅猛提高,促进了新闻的生产,新闻的数量呈现出爆发式增长的态势。这些都表明智能分类、标引方法在新闻领域有着极大的应用价值。此外,中文文本不同于英文文本,词与词之间没有明显的区分,增加了分词的难度,不同的分词结果甚至影响句子的含义。因此,中文新闻文本分类方法的研究有着重要的意义。
		
		
		\section{相关工作}
		\hspace{2.0em}20世纪90年代之后,主要用机器学习方法解决文本分类问题。而传统的机器学习方法在学习文本的语义特征上表现不佳,深度学习能更好地学习与表达特征,且具有良好的建模能力。2014年,注意力机制模型首次在机器翻译领域被提出\cite{1409.0473},其优化了之前的Encoder-Decoder模型,在解码时选择性地从输入向量序列中挑选一个子集进行进一步处理。与此同时TextCNN模型被提出\cite{1408.5882},它将卷积神经网络(CNN)应用到文本分类任务中,利用多个不同尺寸的卷积核来提取句子中的关键信息,从而能够更好地捕捉局部相关性。Vaswani等人于2017年提出Transformer模型\cite{1706.03762},它用注意力机制代替了循环神经网络(RNN)搭建了整个模型框架,且提出了多头注意力机制方法,并在之后的编码器和解码器大量使用多头注意力机制。进一步地,2018年Google提出BERT模型\cite{1810.04805},使用Transformer模型作为算法主要框架,引入掩码语言模型(MLM)与连贯性判定(NSP)方法预处理目标文本,训练更大规模的数据,使NLP达到了一个全新的高度。

		
		\section{备选实验方案}
		\hspace{2.0em}\setlength{\parindent}{2.1em}总体的实验流程:(1) 读取实验数据;(2) 对文本字符长度以及标签类别进行分析,完成数据预处理工作;(3) 建立中文新闻文本分类模型(包括对中文的分词、Word Embedding等工作);(4) 根据测试集数据的评估情况进一步调整及优化模型。
		\par 其中,在(3) 文本分类模型建立部分,我们计划使用三种在文本分类领域中应用较为广泛的深度学习模型,分别为LSTM、LSTM+Attention和BERT,并计划在实验过程中尝试在这些模型的基础上寻找可改进点。
\subsection{长短期记忆网络(LSTM)}
\hspace{2.0em}LSTM\cite{hochreiter1997long}是一种改进的循环神经网络,它极大地改善了原始循环神经网络易于梯度消失或梯度爆炸的问题,常用于处理和预测具有时序关联的数据。在将中文文本转化为词向量,即可使用LSTM进行文本分类。
\subsection{LSTM+Attention}
\hspace{2.0em}LSTM 在一些任务中仍有其一定的局限性,比如其性能受限于固定长度的向量表示。而Attention打破了这种固定长度向量的限制,并能更好地区分不同信息的重要程度。将LSTM与注意力机制结合使用,可能会提高本任务的分类性能。
\subsection{BERT}
\hspace{2.0em}BERT是一个基于双向Transformer的多层Encoder-Decoder结构的模型,具有较高的分类性能和较强的泛化性能。在实验时可以使用Google开源的Bert中文预训练模型对新闻文本数据进行训练和对模型参数的调整,以完成中文新闻的分类任务。



		
\bibliographystyle{unsrt}
%\bibliographystyle{ACM-Reference-Format}
\bibliography{ref}		
		
		
\end{document}